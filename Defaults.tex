%%%%%%%%%%%%%%%% Extra Packages %%%%%%%%%%%%%%%%

% turns on parkskip package
\Parskip{true} % true/false

% \usepackage{}

%%%%%%%%%%%%%%%%%% Variablen %%%%%%%%%%%%%%%%%%%

% img path
\graphicspath{{img/}}
\svgpath{{img/}}

\Logo{true} % true/false
\SetLogoPath{../res/logos/UHH} % UHH, WAP, DRIP
\SetLogoScale{.30}
\SetLogoPlacement{middle} % left, middle, right
\SetTitlespacing{1.5cm} % from the top

\SetAnzahlStudenten{3}


% Student 1
\SetVornameEins{}
\SetNachnameEins{}
\SetMatrikelnummerEins{}

% Student 2
\SetVornameZwei{Paul}
\SetNachnameZwei{Bartel}
\SetMatrikelnummerZwei{7428530}

% Student 3
\SetVornameDrei{}
\SetNachnameDrei{}
\SetMatrikelnummerDrei{}

% Student 4
\SetVornameVier{\ldots}
\SetNachnameVier{\ldots}
\SetMatrikelnummerVier{1234567}

% Student 5
\SetVornameFuenf{\ldots}
\SetNachnameFuenf{\ldots}
\SetMatrikelnummerFuenf{1234567}

% dyanmic, vertical, horizontal, list
\SetNameLayout{dynamic}

\SetModulKurz{Computer Networks}
\SetModulLang{Computer Networks}

\SetFontsize{11} % 10,11,12

% 1 = \singlespacing;
% 2 = \onehalfspacing;
% 3 = \doublespacing;
% 4 = \setstretch{}; [] for strech value  
\SetSpacing[1.2]{4}

\raggedbottom

% A = \Alph*
% a = \alph*
% I = \Roman*
% i = \roman*
% 1 = \arabic*
\SetDefaultLabel{\Alph*)}

% All light themed
%
% Crab, TokyoNight, Solarized, Gruvbox
\SetColors{Solarized}

% default, FiraMath, Gyrebonum, Gyredejavu, Gyrepagella,
% Gyreschola, Gyretermes, STIXTwo, XITS
\SetFontMath{FiraMath}

% default, FiraGo, IBMSans, MerriweatherSans, NotoSans
% SourceSans3
\SetFontSans{FiraGo}

% default, IBMSerif, MerriweatherSerif, NotoSerif
\SetFontSerif{default}

% default, FiraCode, IBMMono, JetBrainsMono, SourceCodePro
\SetFontMono{FiraCode}

\AllSans{true} % true/false

%%%%%%%%%%%%%%%% Package Config %%%%%%%%%%%%%%%%


\epigraphsize{\normalsize}
\setlength \epigraphwidth {\linewidth}
\setlength \epigraphrule {0pt}
\AtBeginDocument{\renewcommand {\epigraphflush}{center}}
\renewcommand {\sourceflush} {center}


\makeatletter
\patchcmd{\epigraph}{\@epitext{#1}}{\itshape\@epitext{#1}}{}{}
\makeatother

\geometry{
	width=16cm, headheight=15pt, a4paper, bottom=1in, top=1in
}

% listings
\lstset{
	backgroundcolor=\color{c_background},              % choose the background color; you must add \usepackage{color} or \usepackage{xcolor}; should come as last argument
	basicstyle=\ttfamily\small\color{c_foreground},    % the size of the fonts that are used for the code
	breakatwhitespace=false,                           % sets if automatic breaks should only happen at whitespace
	breaklines=true,                                   % sets automatic line breaking
	captionpos=b,                                      % sets the caption-position to bottom
	commentstyle=\itshape\color{c_comment},            % comment style
	% deletekeywords={...},                            % if you want to delete keywords from the given language
	escapeinside={\%*}{*)},                            % if you want to add LaTeX within your code
	% extendedchars=true,                              % lets you use non-ASCII characters; for 8-bits encodings only, does not work with UTF-8
	% firstnumber=1000,                                % start line enumeration with line 1000
	% frame=single,	                                   % adds a frame around the code
	keepspaces=true,                                   % keeps spaces in text, useful for keeping indentation of code (possibly needs columns=flexible)
	keywordstyle=\color{c_keyword},                    % keyword style
	% language=Octave,                                 % the language of the code
	% morekeywords={*,...},                            % if you want to add more keywords to the set
	numbers=left,                                      % where to put the line-numbers; possible values are (none, left, right)
	numbersep=5pt,                                     % how far the line-numbers are from the code
	numberstyle=\scriptsize\color{c_foreground},       % the style that is used for the line-numbers
	rulecolor=\color{black},                           % if not set, the frame-color may be changed on line-breaks within not-black text (e.g. comments (green here))
	showspaces=false,                                   % show spaces everywhere adding particular underscores; it overrides 'showstringspaces'
	showstringspaces=false,                            % underline spaces within strings only
	showtabs=false,                                    % show tabs within strings adding particular underscores
	stepnumber=1,                                      % the step between two line-numbers. If it's 1, each line will be numbered
	stringstyle=\color{c_string},                      % string literal style
	tabsize=2,	                                       % sets default tabsize to 2 spaces
	title=\lstname,                                    % show the filename of files included with \lstinputlisting; also try caption instead of title
	caption=
}

% hyperlink 
\hypersetup{
	colorlinks=true,
	linkcolor=blue,
	urlcolor=blue
}
\urlstyle{sf}

% tiks 
\usetikzlibrary{automata, positioning, shadows, arrows}
\tikzset{
	->,
	>=stealth',
	node distance=2.5cm,
	every state/.style={semithick, minimum width=8pt,  fill=backcolour, inner sep=0pt},
	initial text=$ $,
}

%%%%%%%%%%%%%%%% Initialization %%%%%%%%%%%%%%%%

\initialize