\Aufgabenstellung%@@@@@@@@@@@@@@@@@@@@@@@@@@@@@@@@@@@@@@@@@@@@@@@@@
Wir betrachten ein Paket der Länge L, das von Host A über drei Links zu einem Ziel-Host
B geleitet wird. Diese drei Links sind, wie in Abbildung 1 dargestellt durch zwei Switches
verbunden. $d_i$, $s_i$ und $r_i$ bezeichnen die Link-Länge, die Ausbreitungsgeschwindigkeit
(Propagation Speed) und die Übertragungsrate (Transmission Rate) der Verbindung $i$,
für $i$ = 1, 2, 3. Die Switches verzögern jedes Datenpaket um $d_{proc}$.


\Teilaufgabe%@@@@@@@@@@@@@@@@@@@@@@@@@@@@@@@@@@@@@@@@@@@@@@@@@@@@@@
Formulieren Sie unter der Annahme, dass es keine Warteschlangenverzögerungen
gibt, die gesamte Ende-zu-Ende-Verzögerung für das Paket in Form von $d_i$, $s_i$, $r_i$
($i$ = 1, 2, 3) und L.


\Teilaufgabe%@@@@@@@@@@@@@@@@@@@@@@@@@@@@@@@@@@@@@@@@@@@@@@@@@@@@@@
Angenommen, das Paket ist 1.500 Byte groß, die Ausbreitungsgeschwindigkeit auf
allen drei Links ist 2, 5 108 m/s, die Übertragungsraten aller drei Links sind 2, 5
Mbps, die Verzögerung bei der Paketvermittlung ist 3 ms, die Länge des ersten
Links ist 1.000 km, die Länge des zweiten Links ist 4.000 km und die Länge des
letzten Links ist 1.000 km. Wie hoch ist bei diesen Werten die Ende-zu-Ende-
Verzögerung?






