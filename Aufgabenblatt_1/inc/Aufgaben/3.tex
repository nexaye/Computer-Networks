\Aufgabenstellung%@@@@@@@@@@@@@@@@@@@@@@@@@@@@@@@@@@@@@@@@@@@@@@@@@
Angenommen, die Benutzer teilen sich eine 10 Mbps-Verbindung. Des Weiteren nehmen
wir an, dass jeder Benutzer nur 10 \% der Zeit und wenn dann mit 200 kbit/s sendet:


\Teilaufgabe%@@@@@@@@@@@@@@@@@@@@@@@@@@@@@@@@@@@@@@@@@@@@@@@@@@@@@@
Wie viele Benutzer können unterstützt werden, wenn Leitungsvermittlung verwen-
det wird?


\Teilaufgabe%@@@@@@@@@@@@@@@@@@@@@@@@@@@@@@@@@@@@@@@@@@@@@@@@@@@@@@
Nehmen Sie für den Rest dieser Aufgabe an, dass Paketvermittlung verwendet
wird. Ermitteln Sie die Wahrscheinlichkeit, dass ein Benutzer zu einem bestimmten
Zeitpunkt sendet.


\Teilaufgabe%@@@@@@@@@@@@@@@@@@@@@@@@@@@@@@@@@@@@@@@@@@@@@@@@@@@@@@
ngenommen, es gibt 120 Benutzer. Ermitteln Sie die Wahrscheinlichkeit, dass zu
einem bestimmten Zeitpunkt N Benutzer gleichzeitig senden.
\textbf{Hinweis}: Verwenden Sie die Binomialverteilung.
\Teilaufgabe%@@@@@@@@@@@@@@@@@@@@@@@@@@@@@@@@@@@@@@@@@@@@@@@@@@@@@@
Ermitteln Sie die Wahrscheinlichkeit, dass 51 oder mehr Benutzer gleichzeitig
senden.


