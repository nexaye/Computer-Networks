\Aufgabenstellung%@@@@@@@@@@@@@@@@@@@@@@@@@@@@@@@@@@@@@@@@@@@@@@@@@
Abbildung 2 zeigt einen HTTP-Server und einen Client. Nehmen wir an, dass die Round-Trip-Time (RTT) zwischen Client und Server 10 ms beträgt.
Die Zeit, die ein Server
benötigt, um ein Objekt auf seinen ausgehenden Link zu übertragen, beträgt 0,75 ms.
Jede andere HTTP-Nachricht, die kein Objekt enthält, hat eine vernachlässigbare (d.h.null) Übertragungszeit. Gehen Sie davon aus, dass der Client HTTP 1.1 und den IF-
MODIFIED-SINCE-Header verwendet. Bei den folgenden Fragen können Sie davon
ausgehen, dass die TCP-Verbindung bereits aufgebaut wurde.


\Teilaufgabe%@@@@@@@@@@@@@@@@@@@@@@@@@@@@@@@@@@@@@@@@@@@@@@@@@@@@@@
Wie viel Zeit verstreicht für die erste Anfrage? Ist das angeforderte Objekt in der
Antwortnachricht enthalten?


\Teilaufgabe%@@@@@@@@@@@@@@@@@@@@@@@@@@@@@@@@@@@@@@@@@@@@@@@@@@@@@@
ngenommen, die Datei wurde seit der ersten Anfrage \textbf{NICHT} geändert, wie viel
Zeit verstreicht dann für die zweite Anfrage? Ist das angeforderte Objekt in der
Antwortnachricht enthalten?


\Teilaufgabe%@@@@@@@@@@@@@@@@@@@@@@@@@@@@@@@@@@@@@@@@@@@@@@@@@@@@@@
Nehmen wir nun an, der Client stellt wieder 70 Anfragen, eine nach der anderen.
Dabei wartet er auf die Antwort auf eine Anfrage, bevor er die nächste Anfrage
sendet. Nehmen wir an, 40\% der angeforderten Objekte haben sich NICHT geändert,
seit der Client sie heruntergeladen hat. Wie viel Zeit vergeht zwischen dem
Senden der ersten Anfrage durch den Client und dem Abschluss der letzten Anfrage?






